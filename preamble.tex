\begin{comment}
\begin{center}
    \textbf{Vorbemerkung}

    (Version: 2021-05-12-IMIS LaTex Version by Philipp Bzdok)
\end{center}
Anbei ein kommentiertes Template für Projekt- und Abschlussarbeiten in der Medieninformatik. Die grünen Kommentare in jedem Fall vor einer (Zwischen-) Abgabe entfernen (alle!).

Alle Angaben sind eigene Einschätzung (bzw. aus den zitierten Quellen, die aber auch wieder nach eigenen Überlegungen von mir ausgewählt wurden) und entsprechend ohne Gewähr. Nordstrom's Policy gilt auch hier: "Use your own best judgment at all times.".

Es ist ihre Arbeit. Sie investieren 6 Monate Ihrer Lebenszeit in die Arbeit und Sie müssen Sie vor anderen verteidigen. Sehen Sie Hinweise von anderen als Ratschläge: Ernst nehmen, aber immer für sich selbst entscheiden, was passt, was nicht und was auf ein Problem hinweist, was man dann aber anders löst.
Das letzte Wort (v.a. über formelle Aspekte) hat allerdings immer der jeweilige Betreuer.

Zunächst kommen ein paar allgemeine Vorbemerkungen, die für die ganze Arbeit relevant sind (Allgemeine Punkte der Arbeit). Danach die Gliederung einer normalen Arbeit mit Anmerkungen in den entsprechenden Kapiteln bzw. Abschnitten.

Viel Erfolg beim Schreiben.

Daniel Wessel

\textbf{Rundumschlag was wissenschaftliches Arbeiten betrifft:}\\ \url{https://www.youtube.com/watch?v=_ID7q7pXzyc}

\textbf{Allgemeine Punkte der Arbeit}
\begin{itemize}
    \item \textbf{Änderbarkeit von Struktur und Inhalten:} Je nach konkretem Thema kann eine andere Struktur sinnvoll sein. Dies ist ins-besondere bei der Reihenfolge der Analyse-Abschnitte der Fall, kann aber auch ganze Kapitel betreffen. Das ist insbesondere bei Masterarbeiten der Fall, die sich je nach Ausrichtung (z. B. theoretische Arbeit) stark von der sonst üblichen Form unterscheiden. Diese Struktur so früh wie möglich mit dem Betreuer klären, ihm dann per eMail zuschicken und sich kurz bestätigen lassen (generell hat man viel zu tun und erinnert sich nicht unbedingt an alle Absprachen, deswegen einfach im Anschluss kurz eine Zusammenfassung des Gesprächs per eMail schicken).
    \item \textbf{Exposé (Bachelor-/Masterarbeit) bzw. Pflichtenheft (Bachelor-/Masterprojekt) setzt die inhaltliche Bewertungsgrundlage:}  Darin haben Sie (Arbeit) oder Ihr Betreuer (Projekt) festgelegt, was zu erreichen ist. Entsprechend genau überlegen, was man verspricht bzw. zu was man sich verpflichtet.
    \item \textbf{Zielgruppe:} Schreiben Sie die Arbeit so, dass andere Personen verstehen, was Sie machen — auch wenn sie keine (Medien-)Informatik studiert haben. Sprich: Beginnen Sie breit um den Einstieg zu erleichtern und formulieren Sie es allgemein verständlich (z. B. das die öffentliche Verwaltung zunehmend digitalisiert wird) und fassen Sie am Ende eines Kapitels die Punkte allgemeinverständlich wieder zusammen. Dazwischen können Sie auf ein Detaillevel runtergehen und eine Komplexität nutzen, die ein Laie nicht mehr versteht. Da der Text in Absätzen aufgebaut ist, kann der Laie (oder nicht interessierte) diese Absätze überspringen. Der Laie sollte aber zumindest im Prinzip verstehen, was Sie gemacht haben.
    \item \textbf{Roter Faden:} Die Arbeit logisch aufeinander aufbauen: In der Einleitung legen Sie dar, was Sie erreichen wollen und zeigen dabei auch subtil, warum das wichtig / interessant / relevant ist (\textit{"Why should I care?"}). Das geschieht über Belege und Argumente, wann immer Sie "wichtig", "interessant" oder "relevant" im Text verwenden sind Sie dabei gescheitert. \textbf{Alle} Kapitel nach der Einleitung zeigen \textbf{logisch aufeinander folgend}, wie der Zweck der Arbeit erreicht wird.
    \item \textbf{Zwischenstufen-Denken:} So schreiben, dass am Ende jedes Kapitels (Einleitung, Analyse, Konzeption, Realisierung, Dialogbeispiele, Evaluation) der Leser stoppen könnte und die Arbeit selbst fortführen könnte (z. B. nach der Einleitung sich für eine anderes Analysevorgehen entscheiden, oder auf Basis der Analyse eine andere Lösung konzipieren).
    \item \textbf{Eigenes Fazit am Ende jeden Kapitels — Was bedeuten die Ergebnisse für die Arbeit bzw. deren Ziel?} Z. B. am Ende der Analyse kurz zusammenfassen, was die wesentlichen Punkte für die weitere Entwicklung sind — mit Fokus auf das nächste Kapitel (hier: Konzeption). Am Ende der Konzeption kurz zusammenfassen, was die wesentlichen Punkte für die weitere Entwicklung (jetzt: Realisierung) sind, etc. pp. Ein gutes Fazit ist viel Arbeit und setzt ein gut geschriebenes Kapitel voraus.
    \item \textbf{Professionelle Zwischenabgaben:} Wenn Sie den Text an den Betreuer geben, gehen Sie vorher kritisch drüber. Wenn Textmarken falsch gesetzt sind, die Formatierung zusammengebrochen ist, etc. dann muss sich der Betreuer da erst mal durchwühlen um zum Inhalt zu kommen. Gute Formatierung macht keine schlechte Arbeit gut (die inhaltlichen Fehler fallen eher umso deutlicher auf), aber schlechte Formatierung macht eine gute Arbeit schlechter.
    \item \textbf{Professionell und offen Kommunizieren:} Weder sich selbst kreuzigen noch Fehler verbergen, sondern berichten, was gemacht wurde und mit Fehlern konstruktiv umgehen.
\end{itemize}

\textbf{Sprache: Stil}
\begin{itemize}
    \item \textbf{Kein Ich, keine Hero's Journey:} Es ist — im Prinzip — egal, wer die Arbeit durchgeführt hat (zumindest für die Qualität der Arbeit, nicht für die Bewertung Ihrer Leistung). Was überzeugen muss ist das Vorgehen, die Belege und Argumentation. Entsprechend stellen Sie das Vorgehen neutral dar ohne auf sich selbst zu verweisen (eher passiv verwenden). Ausnahmen sind u.a. in Danksagung, Widmung, Eidesstattliche Erklärung.
    \item \textbf{Meta vermeiden}: Sie müssen an vielen Stellen darauf hinweisen was kommt (z. B. zu Beginn eines jeden Kapitels). Reißen Sie den Leser aber dabei nicht aus dem Text. Mental ist der Leser dabei, ihre Arbeit zu beobachten, was Sie konkret getan haben. Wenn Sie ihm jetzt sagen "In diesem Kapitel ..." dann ziehen Sie ihn aus dem Text und bringen ihn dazu, über den Text nachzudenken statt über das, was gemacht wurde. Bleiben Sie bei dem, was Sie gemacht haben, z. B. "Nachfolgend werden ...". 
    \item \textbf{Kapitel $\neq$ Unterkapitel $\neq$ Abschnitt:} 1 ist ein Kapitel, 1.1 ein Unterkapitel, 1.1.1 ein Abschnitt, 1.1.1.1 existiert nicht. Wenn Sie noch mehr Einteilungen brauchen, dann verwenden Sie Abschnitte mit Fettdruck zu Beginn (wie in diesem Abschnitt, dann aber ohne die Bulletpoints).
    \item \textbf{Kein Bulletpoint-Text:} Bulletpoints sind nur an wenigen Stellen hilfreich, z. B. bei Aufzählungen. Sätze nie mit Bulletpoints aufzählen. Entweder die Sätze auf Stichworte reduzieren oder eine Tabelle draus machen.
    \item \textbf{Auch digital gedrucktes ist tot und macht nichts mehr:} Sie berichten was Sie gemacht haben — außerhalb des Berichtes. Entsprechend nie schreiben, dass z. B. in der Analyse der Sachverhalt analysiert wird. Der Bericht macht von sich aus nichts. Sie stellen dort die Ergebnisse der Analyse dar.

    \item \textbf{Umgangssprache vermeiden, Hochgestochene Sprache vermeiden:} Weder Umgangssprache (Sozialpädagogensprache; "tut", "Das Ganze ...", "etwas für ihre Gesundheit zu machen") noch Hochgestochen (à la Philosophendeutsch) schreiben. Wissenschaftliche Sprache ist nach \textcite{alley_1996}:
    \begin{itemize}
        \item präzise: sagen was man meint (richtige Wort, richtiges Detaillevel)
        \item klar: vermeiden Sachen zu sagen/implizieren, die man nicht meint, d.h. Ambiguität und unnötige Komplexität (v.a. in der Wahl der Wörter) vermeiden
        \item ehrlich: direkt und offen kommunizieren
        \item prägnant: jedes Wort sollte zählen
        \item bekannt/vertraut: neue Fakten in bekannten Kontext verankern
        \item flüssig: von Satz zu Satz, Absatz zu Absatz, ohne dass der Leser stolpert
    \end{itemize}
    \fullcite{alley_1996}
\end{itemize}

\textbf{Sprache: Zeitformen}
\begin{itemize}
    \item \textbf{In einem Bericht berichten Sie:} Entsprechend Vergangenheitsform verwenden. Sie berichten über eine abgeschlossene Arbeit, selbst wenn diese noch läuft als Sie es geschrieben haben, und selbst bei Zielen der Arbeit. Ausnahmen sind selten, z. B. bei den Ergebnissen ("Daten zeigen ..." — sie machen es ja noch) und Aus-blick bzw. Vorschläge für die Zukunft.
\end{itemize}

\textbf{Sprache: Absätze}
\begin{itemize}
    \item \textbf{Eine Sinneinheit = 1 Absatz:} Absätze behandeln immer einen Punkt, eine Sinneinheit. Wenn eine halbe Seite lang kein Absatz verwendet wird, liegt meist ein Problem vor.
    \item \textbf{Absätze sind immer länger als ein Satz:} Keine Einsatz-Stückel-Absätze. Einzige Ausnahme: Sie wollen, dass der Leser die komplette Aufmerksamkeit auf diesen zentralen Satz lenkt. Das kann man 1-2 Mal in einer Arbeit machen.
\end{itemize}

\textbf{Interne Verweise}
\begin{itemize}
    \item \textbf{Verweise statt Wiederholungen:} Üblicherweise braucht man einen Sachverhalt nur ein Mal zu beschrieben — dann verweist man an anderer Stelle auf den konkreten Abschnitt. Das ist auch der Grund für die Nummerierung — es erlaubt Ihnen, den Leser präzise zu den Punkt in der Arbeit zu schicken, an dem Sie auf den Sachverhalt eingehen. Nutzen Sie also "wie in der Kontextanalyse (2.4) beschrieben ...", dann weiß der Leser, wie lange er blättern muss.
\end{itemize}

\textbf{Zitationen}
\begin{itemize}
    \item \textbf{APA verwenden:} American Psychological Association (7. Ausgabe) Stil verwenden. Gibt genug Informationsseiten dazu im Netz und Literaturmanager können diesen Stil üblicherweise.
    \item \textbf{Richtige Zitationen:} Falsche "ich füg die richtigen Zitationen später ein" Ab-gaben verbrennen Ihnen den Betreuer. Arbeiten Sie von Anfang an mit einem Literaturverwaltungsprogramm, in dem die verwendeten Quellen richtig eingetragen sind (nächster Punkt).
    \item \textbf{Autorennamen müssen genannt werden, aber nicht hervorheben:} Die Wissenschaft sollte keinen Personenkult kennen — das können gerne Religionen oder Ideologien übernehmen. Wer was herausgefunden hat, ist egal. Die Qualität der Arbeit zählt. Entsprechend nicht "Die Autoren xyz haben herausgefunden das ABC vor-liegt" sondern "Da ABC vorliegt (Autoren Jahr) ...".
    \item \textbf{Kein Paper-Denglish:} Ja, im englischen heißt es Paper. Im Text sind es aber Artikel oder Konferenzbeiträge oder Buchkapitel oder was auch immer. Üblicherweise muss man den Typ auch nicht erwähnen (im Normalfall wurde man eh Artikel/Konferenzbeiträge zitieren). Bitte nicht so was wie "Im Paper von xyz ...", das klingt nach Möchtegern-coole Manta mit Fuchsschwanz Sprechweise. Ein-fach berichten wie die Befundlage ist, über Autorenname und Jahr (meist in Klammern) belegen wo es herkommt (sonst ein Plagiat) und die Belege und Argument in der Arbeit sprechen lassen.
    \item \textbf{Wörtliche Zitate nur wenn es nicht anders geht:} In den meisten Fällen geben Sie Befunde oder Argumente mit Ihren eigenen wieder (mit Quellenangabe). Wörtliche Zitate braucht man nur in sehr seltenen Fällen. Z. B. treffende Aussage von Evaluationsteilnehmern, oder eine Definition, die man 1:1 so sagen muss.
    \item \textbf{Fußnoten vermeiden:} Entweder es ist wichtig genug, genannt zu werden, oder es ist so unwichtig, dass es raus kann. Fußnoten reißen den Leser aus dem Text. Einzige Ausnahme: Bei der ersten Verwendung des generische Maskulinums.
    \item \textbf{Es gibt mehr Plagiate als nur Quellen nicht angeben:} Es gibt z. B. Übersetzungsplagiate, bei denen Sie einen Text(teil) einfach auf deutsch übersetzen ohne die Quelle anzugeben, oder dass man einfach die Argumentstruktur und Quellenangaben aus einem anderen Text übernimmt (ohne die Quelle anzugeben). Sie schmücken sich dann mit fremden Federn und Blender sind selten willkommen.
    \item \textbf{Literaturverwaltungsprogram nutzen:} Literaturverwaltungsprogramm (z. B. Zotero) hilft extrem bei der richtigen Zitierung, aber ACHTUNG: Wenn die Angabe in Zotero fehlt, ist auch die automatische Generierung des APA Stils falsch! GIGO gilt auch hier.
\end{itemize}

\textbf{Abbildungen und Tabellen}
\begin{itemize}
    \item Expliziter Verweis vom Text auf die Abbildung/Tabelle immer im Absatz vor der Abbildung/Tabelle (Leser stolpern über Abbildung/Tabelle, suchen dann nach oben nach mehr Informationen).
    \item Auf \textbf{Lesbarkeit} achten! Schriftgröße und Auflösung (bei Bildern) im Probedruck überprüfen!
    \item Für Schwarz/Weiß-Druck und für Farbfehlsichtige geeignet.
    \item Tabellen nach \textbf{APA Stil} (nur horizontale Linien und nur nach Header oder vor Footer.
\end{itemize}

\textbf{Druck}
\begin{itemize}
    \item Arbeit einseitig drucken!
    \item \textbf{PDF Druck und Suchfunktion:} Zuerst als PDF drucken, dann nach "Fehler! Verweisquelle konnte nicht gefunden werden." suchen. Word bricht schon mal gerne die Verlinkungen und das sieht man erst im Druck! Generell PDF Dokument kritisch durchgehen und PDF auch zum Drucken im Copyshop verwenden!
    \item Nach dem Druck und vor dem Binden alle Seiten selbst sowie von einer anderen Person durchgehen (lassen).
\end{itemize}
\end{comment}
\newpage
\chapter*{Danksagung}
\begin{comment}
Kurzer Dank an Personen, die Sie bei der Arbeit unterstützt haben. Z. B. inoffizielle Betreuer, Teilnehmer an den Evaluationen (nie namentlich nennen), Medientechnik, Sekretärin, etc. pp. — nur wenn Sie den Personen wirklich dankbar sind. (Ist nett aber für die Bewertung irrelevant.)
Falls nicht verwendet diese Seite einfach entfernen.
\end{comment}

Text \dots

\newpage
\chapter*{Kurzfassung}
\begin{comment}
Abstract schon für Zwischenabgabe schreiben. Später kommen dann noch Sätze dazu, aber Grundgerüst steht.

Kein "Teaser" sondern eine kurze Zusammenfassung (das, was man braucht, um sich schnell einen Überblick zu verschaffen, ob es sich lohnt, die Arbeit zu lesen).

Inhalt umfasst die zentralen Punkte aller Kapitel, von Ziel/Fragestellung bis Ausblick.

Nie länger als diese eine Seite (inkl. Schlüsselwörter).
\end{comment}

Text \dots

\section*{Schlüsselwörter}
\begin{comment}
Verwendete Literatur gibt Hinweise auf passende Stichwörter. Das sind die Suchbegriffe, die man bei einer Literatursuche verwenden würde.
\end{comment}

Text \dots

\newpage
\chapter*{Abstract}
\begin{comment}
Englische Version der Kurzfassung.
Nicht einfach Google Translate oder DeepL verwenden. Trifft die Nuancen nicht und klingt z. T. nach Yoda.
\end{comment}

Text \dots

\section*{Keywords}
Text \dots

\newpage