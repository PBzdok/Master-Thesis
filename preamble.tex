\chapter*{Acknowledgement}
\begin{comment}
Kurzer Dank an Personen, die Sie bei der Arbeit unterstützt haben. Z. B. inoffizielle Betreuer, Teilnehmer an den Evaluationen (nie namentlich nennen), Medientechnik, Sekretärin, etc. pp. — nur wenn Sie den Personen wirklich dankbar sind. (Ist nett aber für die Bewertung irrelevant.)
Falls nicht verwendet diese Seite einfach entfernen.
\end{comment}

Text \dots

\newpage
\chapter*{Kurzfassung}
\begin{comment}
Abstract schon für Zwischenabgabe schreiben. Später kommen dann noch Sätze dazu, aber Grundgerüst steht.

Kein "Teaser" sondern eine kurze Zusammenfassung (das, was man braucht, um sich schnell einen Überblick zu verschaffen, ob es sich lohnt, die Arbeit zu lesen).

Inhalt umfasst die zentralen Punkte aller Kapitel, von Ziel/Fragestellung bis Ausblick.

Nie länger als diese eine Seite (inkl. Schlüsselwörter).
\end{comment}

Text \dots

\section*{Schlüsselwörter}
\begin{comment}
Verwendete Literatur gibt Hinweise auf passende Stichwörter. Das sind die Suchbegriffe, die man bei einer Literatursuche verwenden würde.
\end{comment}

Text \dots

\newpage
\chapter*{Abstract}
\begin{comment}
Englische Version der Kurzfassung.
Nicht einfach Google Translate oder DeepL verwenden. Trifft die Nuancen nicht und klingt z. T. nach Yoda.
\end{comment}

Text \dots

\section*{Keywords}
Text \dots

\newpage