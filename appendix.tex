\phantomsection
\addcontentsline{toc}{chapter}{Abbildungsverzeichnis}
\listoffigures
\clearpage

\phantomsection
\addcontentsline{toc}{chapter}{Tabellenverzeichnis}
\listoftables
\clearpage

\phantomsection
\addcontentsline{toc}{chapter}{Quellen}
\chapter*{Quellen}

\phantomsection
\printbibliography[heading=subbibintoc, nottype=online, nottype=software]

\phantomsection
\printbibliography[heading=subbibintoc, type=online, title=Websites]
\begin{comment}
Websites haben nicht die gleiche Fundierung/Überzeugungskraft wie wissenschaftliche Literatur (peer-review/Anspruch). Die Websites separat aufzuführen, macht es einfacher zu sehen, was wirklich wissenschaftlich fundierte Literatur ist.
\end{comment}

\phantomsection
\printbibliography[heading=subbibintoc, type=software, title=Software]
\begin{comment}
Software-Name mit Versionsnummer und Link zur Website. Nur was für die konkrete Arbeit relevant ist. Das Sie die Arbeit mit Word geschrieben haben, ist irrelevant. Das sieht man. LaTeX auch.
\end{comment}
\clearpage

\phantomsection
\addcontentsline{toc}{chapter}{Abkürzungen}
\chapter*{Abkürzungen}
\begin{comment}
Kann recht kurz ausfallen, aber falls Sie bestimmte Abkürzungen häufiger verwenden, hier aufführen. Geht nur darum, dass der Leser hierhin blättern könnte, wenn er über eine unbekannte Abkürzung stolpert. Ist beim digitalen Lesen hinfällig geworden. Hier geht es wirklich nur um Akronyme, z. B. ICBM = Intercontinental Ballistic Missile, oder PIN = Personal Identification Number. Die Erläuterung von Begriffen erfolgt im Glossar.
\end{comment}

\begin{description}
    \item [\textbf{Akronym}] Ausgeschriebenes Akronym
\end{description}
\clearpage

\phantomsection
\addcontentsline{toc}{chapter}{Glossar}
\chapter*{Glossar}
\begin{comment}
Unterschiedliche Disziplinen verwenden z. T. die selben Begriffe für unterschiedliche Sachverhalte und unterschiedliche Begriffe für die selben Sachverhalte. Hier können Sie wiederkehrende zentrale Begriffe der Arbeit kurz definieren (z. B. Adipositas, Depression, Legasthenie, etc.). Früher wichtiger, dann konnte man hierhin blättern und musste nicht den ganzen Text absuchen. Heute verwendet man digital die Suchfunktion. Trotzdem ernst nehmen.
\end{comment}

\begin{description}
    \item [\textbf{Begriff}] Kurze Erläuterung
\end{description}
\clearpage

\phantomsection
\addcontentsline{toc}{chapter}{Anhänge}
\chapter*{Anhänge}
\begin{comment}
Zusätzliche Informationen die zu lang für die Arbeit sind können hier verfügbar gemacht werden.

Aber auch an die DVD denken — was ist dort besser aufgehoben? Die Zeiten, in denen man Programmcode manuell eingetippt hat, sind ja glücklicherweise lange vorbei, deswegen macht Code hier wenig Sinn.

Ist entsprechend ein Priorisierung: Was würde sich der Leser vielleicht gerne während des Lesens der Arbeit (z. B. im Zug) ansehen, wenn er auch gerade nicht auf die DVD zugreifen kann (kein DVD Laufwerk)?

Inhalte sind oft: Überblick der Inhalte der DVD, Fragebögen (falls digital Screenshots oder neu für den Druck formatiert), Interviewleitfäden, etc. Selten detailliertere Evaluationsergebnisse.

Hier kurz die Zwischenüberschriften nennen und evtl. 1 Satz, was dort zu finden ist (falls es nicht schon durch die Zwischenüberschrift klar ist). 
\end{comment}

Text \dots

\phantomsection
\addcontentsline{toc}{section}{Anhang A: Inhalt der DVD}
\section*{Anhang A: Inhalt der DVD}
\begin{comment}
Oft ein Default: Was findet man auf der beiliegenden DVD in welchem Verzeichnis? Max. 1 Seite.

\textbf{In jedem Fall} die PDF der Arbeit, den Programmcode, Daten (anonymisiert!).

\textbf{Niemals} Interviewaufzeichnungen, Einverständniserklärungen oder ähnliche personenbezogene Daten auf die DVD brennen — Sie haben in den meisten Fällen Anonymität zugesichert und die DVD ist frei zugänglich (ein Exemplar der Arbeit kommt in die Bibliothek). 
\end{comment}

Text \dots

\phantomsection
\addcontentsline{toc}{section}{Anhang B: Anhangstitel}
\section*{Anhang B: Anhangstitel}
\begin{comment}
Weitere Inhalte je nachdem, wo der Leser ohne großen Aufwand hinspringen sollte.
\end{comment}

Text \dots

\clearpage

\phantomsection
\addcontentsline{toc}{chapter}{Erklärung}
\chapter*{Erklärung}
Ich versichere an Eides statt, die vorliegende Arbeit selbstständig verfasst und nur die angegebenen Quellen benutzt zu haben.

\begin{comment}
[Nach Ausdruck unterschreiben. Muss auf Papier sein.]
\end{comment}

-----------------------------------------------------------------

Lübeck, \today, \authorMA